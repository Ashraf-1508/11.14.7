\let\negmedspace\undefined
\let\negthickspace\undefined
\documentclass[journal,12pt,twocolumn]{IEEEtran}
\usepackage{cite}
\usepackage{amsmath,amssymb,amsfonts,amsthm}
\usepackage{algorithmic}
\usepackage{graphicx}
\usepackage{textcomp}
\usepackage{xcolor}
\usepackage{txfonts}
\usepackage{listings}
\usepackage{enumitem}
\usepackage{mathtools}
\usepackage{gensymb}
\usepackage{caption}
\usepackage[breaklinks=true]{hyperref}
\usepackage{tkz-euclide} % loads  TikZ and tkz-base
\usepackage{listings}
\usepackage{gvv}
\newtheorem{theorem}{Theorem}[section]
\newtheorem{problem}{Problem}
\newtheorem{proposition}{Proposition}[section]
\newtheorem{lemma}{Lemma}[section]
\newtheorem{corollary}[theorem]{Corollary}
\newtheorem{example}{Example}[section]
\newtheorem{definition}[problem]{Definition}
%\newtheorem{thm}{Theorem}[section] 
%\newtheorem{defn}[thm]{Definition}
%\newtheorem{algorithm}{Algorithm}[section]
%\newtheorem{cor}{Corollary}
\newcommand{\BEQA}{\begin{eqnarray}}
\newcommand{\EEQA}{\end{eqnarray}}
\newcommand{\define}{\stackrel{\triangle}{=}}
\theoremstyle{remark}
\newtheorem{rem}{Remark}

%\bibliographystyle{ieeetr}
\begin{document}
%

\bibliographystyle{IEEEtran}


\vspace{3cm}

\title{
%  \logo{
Assignment-2

\large{EE:1205 \brak{Signals  Systems}}

Indian Institute of Technology, Hyderabad
%  }
}
\author{Md Ayaan Ashraf

EE23BTECH11041
}  
\maketitle
\newpage
\bigskip
\renewcommand{\thefigure}{\arabic{figure}}
\renewcommand{\thetable}{\arabic{table}}
%\renewcommand{\theequation}{\theenumi}
\section*{\textit{\textbf{Question}}}
The motion of a particle executing simple harmonic motion is described by the
displacement function, $x(t)$ = $A$ $cos$ ($\omega$$t$ +$\phi$).
If the initial $(t = 0)$ position of the particle is $1 cm$ and its initial velocity is $\omega\quad cm/s$, what are its amplitude and initial phase angle ? The angular frequency of the particle is $\pi\quad s^{-1}$. If instead of the cosine function, we choose the sine function todescribe the SHM : $x$ = $B$ $sin$ ($\omega$$t$ +$\alpha$), what are the amplitude and initial phase of the
particle with the above initial conditions.
\end{document}
